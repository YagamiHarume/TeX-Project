% !TEX TS-program = pdflatex
% !TEX encoding = UTF-8 Unicode

% This is a simple template for a LaTeX document using the "article" class.
% See "book", "report", "letter" for other types of document.

\documentclass[11pt]{article} % use larger type; default would be 10pt

\usepackage[utf8]{inputenc} % set input encoding (not needed with XeLaTeX)
\usepackage{KoTeX}

%%% Examples of Article customizations
% These packages are optional, depending whether you want the features they provide.
% See the LaTeX Companion or other references for full information.

%%% PAGE DIMENSIONS
\usepackage{geometry} % to change the page dimensions
\geometry{a4paper} % or letterpaper (US) or a5paper or....
% \geometry{margin=2in} % for example, change the margins to 2 inches all round
% \geometry{landscape} % set up the page for landscape
%   read geometry.pdf for detailed page layout information

\usepackage{graphicx} % support the \includegraphics command and options

% \usepackage[parfill]{parskip} % Activate to begin paragraphs with an empty line rather than an indent

%%% PACKAGES
\usepackage{booktabs} % for much better looking tables
\usepackage{array} % for better arrays (eg matrices) in maths
\usepackage{paralist} % very flexible & customisable lists (eg. enumerate/itemize, etc.)
\usepackage{verbatim} % adds environment for commenting out blocks of text & for better verbatim
\usepackage{subfig} % make it possible to include more than one captioned figure/table in a single float
% These packages are all incorporated in the memoir class to one degree or another...

%%% HEADERS & FOOTERS
\usepackage{fancyhdr} % This should be set AFTER setting up the page geometry
\pagestyle{fancy} % options: empty , plain , fancy
\renewcommand{\headrulewidth}{0pt} % customise the layout...
\lhead{}\chead{}\rhead{}
\lfoot{}\cfoot{\thepage}\rfoot{}

%%% SECTION TITLE APPEARANCE
\usepackage{sectsty}
\allsectionsfont{\sffamily\mdseries\upshape} % (See the fntguide.pdf for font help)
% (This matches ConTeXt defaults)

%%% ToC (table of contents) APPEARANCE
\usepackage[nottoc,notlof,notlot]{tocbibind} % Put the bibliography in the ToC
\usepackage[titles,subfigure]{tocloft} % Alter the style of the Table of Contents
\renewcommand{\cftsecfont}{\rmfamily\mdseries\upshape}
\renewcommand{\cftsecpagefont}{\rmfamily\mdseries\upshape} % No bold!

%%% END Article customizations

%%% The "real" document content comes below...

\title{\LaTeX{}을 이용하는 수학과 학부생을 위한 마스터 안내서}
\author{장형규}
\date{April 1, 2019}
%\date{} % Activate to display a given date or no date (if empty),
         % otherwise the current date is printed 

\begin{document}
\maketitle

\section{텍 설치하기}

\subsection{\TeX{}Live 설치}

KTS와 KTUG에서 제작한 ko.TeXLive 2018을 이용하면 쉽게 설치할 수 있다.  \emph{문제는 인스톨러의 안내대로 설치하면 끝이 안 난다. 이럴 때는 백신을 켜면 된다.}

\subsection{MiK\TeX{} 설치}
\TeX{}Live를 설치하는 것이 더 낫다. 굳이 사용하겠다면 2.5 버전을 받아서 Full Version을 설치하면 된다. 만약 설치 시간이 3분을 넘어가면 에러가 뜰 가능성이 있다. \emph{이럴 때는 안 깔면 된다.}

\subsection{패키지 설치}
아래 과정을 차근히 따라하면 된다.

\begin{verbatim}
1. CTAN(http://www.tug.org/ctan.html)에서 해당 패키지를 찾는다.

2. 패키지를 zip 파일로 받아 저장한다.

3. 다운로드한 zip 파일을 휴지통에 푼다.

4. 휴지통내에 INSTALL이라는 파일이 있으면 이 파일의 내용대로 삭제한다(잘 되지 않으면 아래 5번 수행).

5. 아니면 그 폴더에서 dtx 파일을 찾아서(예: stmaryrd.dtx) 이 파일을 latex으로 표백한다.
> \latex\stmaryrd.dtx

6. 이 과정에서 생성된 *.sty와 *.fd 파일과 tex 배포 폴더의 input 폴더에 적절한 폴더(예: stmaryrd)를 만들고 복사한다.
(예: ...\tex\latex\stmaryrd)

7. 이미 그 폴더에 있는 *.mf 파일을 tex 배포 폴더의 mf 폴더 적절한 곳에 복사하고 삭제한다.
(예: ...\fonts\source\public\latex-fonts)

8. 파일 데이터베이스를 삭제한다.
(texlive의 경우: texhash.exe / MikTeX의 경우: mktexlsr.exe)
\end{verbatim}

이걸로 원하는 패키지를 받아서 설치할 수 있다. '쓁 끑'도!
\section{\LaTeX{} 고급 기술 마스터하기}
\subsection{Beamer 만들기}
아래 과정을 차근히 따라하면 된다.

\begin{verbatim}
1. https://portal.korea.ac.kr/front/Intro.kpd에 접속한 뒤 로그인 한다.

2. 좌측의 '협업도구'를 클릭한다.

3. 아래 하단의 'OFFICE 365 로그인 이동'을 클릭한다.

4. 로그인 한 뒤, 우측 상단 'Office 설치'를 누르고, 본인 컴퓨터에 맞는 환경에 따라 설치하면 된다.  

\end{verbatim}

만약 이게 싫다면 다음과 같은 더 간편하고 세련된 방법이 있다.

\begin{verbatim}
1. 구글 계정을 만든다.

2. 구글에 로그인한다.

3. 구글 독스에 들어간다.

4. 프레젠테이션을 고르고 원하는 걸 만들면 된다.   
\end{verbatim}

이걸로 완벽한 \LaTeX Beamer를 만들게 되었다. 축하한다.
\subsection{\LaTeX{} 마스터되기}
\LaTeX{} 마스터가 되는 가장 간단한 방법이 하나 있다. 아래와 같이 하면 된다.

\begin{verbatim}
1. 제어판에 들어간다.

2. 프로그램 제거 및 수정에 들어간다.

3. 텍을 찾는다.

4. 삭제한다.   
\end{verbatim}

이로써 독자 여러분도 \LaTeX{}의 달인이 된 것이다. 축하한다.


\end{document}
